\documentclass[
8pt, % Set the default font size, options include: 8pt, 9pt, 10pt, 11pt, 12pt, 14pt, 17pt, 20pt
%t, % Uncomment to vertically align all slide content to the top of the slide, rather than the default centered
%aspectratio=169, % Uncomment to set the aspect ratio to a 16:9 ratio which matches the aspect ratio of 1080p and 4K screens and projectors
]{beamer}

\graphicspath{{Images/}{./}}

\usepackage{booktabs}
\usepackage{ctex}

\usetheme{Madrid}

\usefonttheme{default}

\usepackage{palatino}
\usepackage[default]{opensans}

\useinnertheme{circles}

%----------------------------------------------------------------------------------------
%	PRESENTATION INFORMATION
%----------------------------------------------------------------------------------------
\title{参与研究安全操作系统的初步想法}
\institute[Massclouds]{\large 乾云科技}
%----------------------------------------------------------------------------------------

\begin{document}

	%----------------------------------------------------------------------------------------
	%	TITLE SLIDE
	%----------------------------------------------------------------------------------------
	\begin{frame}
		\titlepage
	\end{frame}

	%----------------------------------------------------------------------------------------
	%	OBJECTIVE
	%----------------------------------------------------------------------------------------
	\begin{frame}
		\frametitle{目标}
		{\large 从zCore起步,参与安全操作系统的研究工作:}
		\begin{itemize}
			\item {\large 发展一个面向AIoT领域的安全操作系统,以RISCV为主要目标体系结构,向上提供与Zircon保持同步的系统调用接口 以及 与Linux kernel保持兼容的系统调用接口。}
			\item {\large 以Rust-lang为主要的实现语言,架构上参照Zircon的整体架构设计,具体细节参照Linux kernel(arch/riscv)。}
			\item {\large 通过开源社区的方式组织研究和开发,建立各层面开发者参与协作的通道,建立应用层面使用者的反馈通道。}
		\end{itemize}
		{\large 下面就目前进行的准备工作和对将来工作的设想做一个汇报:}
	\end{frame}

	%----------------------------------------------------------------------------------------
	%	TABLE OF CONTENTS SLIDE
	%----------------------------------------------------------------------------------------
	\begin{frame}
		\frametitle{纲要}
		\tableofcontents
	\end{frame}

	%----------------------------------------------------------------------------------------
	%	PRESENTATION BODY SLIDES
	%----------------------------------------------------------------------------------------
	\section{zCore学习过程和疑问}
	
	\subsection{引导阶段}
	
	\begin{frame}
		\frametitle{1 入口\_start}
		\begin{enumerate}
			\item 通过select\_stack为primary hart查找和配置内核栈,为函数调用作准备
			\begin{block}{}
			建议这里不必按照固定的最大值(MAX\_HART\_NUM)为所有的harts预分配页。此时还不知道Harts的总数。可以只为primary hart在bss中预分配栈,等primary hart初始化过程中确定harts数量之后,再按需分配。启动secondary harts可以改一下,把它们的启动时机推后,具体想法后面会提到。
			\end{block}
			\item 进入primary hart的启动函数primary\_rust\_main
			\item 清零BSS段
			\item 初始化页表,为切换到虚拟空间作准备(见2)
			\item 检查设备树dtb有效性
			\item 启动secondary harts(见3)
			\item 进入下阶段primary初始化
		\end{enumerate}
	\end{frame}

	\begin{frame}
		\frametitle{2 启动阶段页表设置boot\_page\_table}
		\begin{enumerate}
			\item 设置物理地址与虚拟地址相等的跳板页
			\item 从虚存开始地址映射物理空间的前128G
			\item 基于SV39模式启用mmu
			\begin{block}{}
				目前zCore是固定采用sv39的模式。可以借鉴Linux kernel的方式,通过写后读轮流尝试sv57,sv48来自动检测所在硬件平台支持的最大模式。
			\end{block}
			\item 调整栈寄存器sp和返回地址寄存器ra,完成从物理空间到虚拟空间转换
			\item 设置状态寄存器,从此允许kernel访问User页面
			\begin{block}{}
				建议采取保守策略,默认情况下始终不开,即默认不允许内核直接访问用户页面,仅在必须执行kernel/user之间内存拷贝或设置的时候,再临时打开,操作完后随即关闭。
			\end{block}
		\end{enumerate}
	\end{frame}

	\begin{frame}
		\frametitle{3 启动Secondary Harts}
		\begin{enumerate}
			\item 检查sbi是否支持HSM扩展,不支持则退化为单核系统
			\begin{block}{}
				HSM是OpenSBIv0.7之后才支持,可以考虑参考Linux kernel for riscv的处理方式:\\
				(1)这里不检查HSM的支持。\\
				(2)把那个Atomic全局变量STARTED提前到\_start入口位置,用它保证只有一个Hart能通过,其它在STARTED变量spin等待。那个唯一能通过的就是Primary Hart。\\
				(3)Primary Hart启动各secondary harts时,如支持HSM,就如当前的办法用SBI唤醒;不支持的,打开STARTED的封锁。
			\end{block}
			\item 遍历设备树文件dtb,发现primary之外所有的hart,通过SBI启动它们,执行secondary的启动流程(见4)。
			\begin{block}{}
				感觉这里就启动secondary harts有点早了,一是这里需要单独处理dtb,其它几处还有单独处理dtb的地方,代码有点分散;二是启动了后面也没有真正运行,还是要等primary的信号。
				建议由Primary Hart在一处统一处理dtb,生成各种管理结构,包括Harts对象数组,然后为所有secondary harts准备好栈,再kick它们。
			\end{block}
		\end{enumerate}
	\end{frame}

	\begin{frame}
		\frametitle{4 Secondary Hart的启动流程}
		\begin{enumerate}
			\item 设置栈:zCore直接用hartid这个硬件ID作为CPU的编号,用它来对应包括系统栈在内的各种CPU专属资源。
			\begin{block}{}
				直接利用hartid作为CPU编号有点问题。
				按照Riscv规范【The RISC-V Instruction Set Manual - Volume II: Privileged Architecture】3.1.5节:
				至少有一个HART的hartid是零,但并不要求所有hartid是连续的。实现策略由厂商自行决定。
				由于不能保证hartid是从零开始的连续数值区间,所以当索引不合适。建议按照通常做法,维护一个CPU数组,CPU在数组的位置即逻辑ID作为CPU资源的索引。
			\end{block}
			\item zCore占用TP寄存器来维持hartid。
			\begin{block}{}
				对每个hart来说,0号系统寄存器就是harid只读寄存器,不必再占用tp寄存器。tp可以用来保持cpu逻辑ID或者直接保存对应percpu的引用指针。
			\end{block}
			\item 通过循环实现HART对应内核栈的定位。
			\begin{block}{}
				应该用索引+移位实现就可以吧?
			\end{block}
		\end{enumerate}
	\end{frame}

	\begin{frame}
		\frametitle{4 Secondary Hart的启动流程}
		\begin{enumerate}
			\item 启用分页
			\item 逐级进入secondary主函数secondary\_main,最后通过HAL的方法完成初始化
			\item 执行HAL初始化:通过kernel\_hal::secondary\_init
			\item 初始化本Hart的根中断控制器intc
			\item 初始化平台级中断控制器plic
			\begin{block}{}
				把plic放到secondary hart的初始化过程中不合适,如前所述,如果SBI不支持HSM,就不仅是退化到单核的问题,连设备中断也没有了。
				这个plic是平台级的,只需要初始化一次,建议还是由primary hart初始化为妥。
			\end{block}
		\end{enumerate}
	\end{frame}

	\begin{frame}
		\frametitle{5 primary\_main主函数}
		\begin{block}{}
			建议用一个Context包装原本的全局变量,在此处定义后(是局部变量),作为参数一直传递下去,直至准备启动secondary harts
			在准备启动secondary harts前,释放掉不再需要的,其它的用ARC和Mutex封装后clone,传递给那些secondary harts
		\end{block}
		\begin{enumerate}
			\item 日志初始化
			\item 内存初始化
			\item HAL层对primary hart的早期初始化
			\item 把从dtb中发现的可用内存范围加入到系统堆中
			\item HAL层对primary hart的正式初始化
			\item 唤醒所有处于等待状态的secondary harts
			\begin{block}{}
				primary hart在唤醒secondary hart之前应该先使用内存屏障,毕竟secondary将要使用的内存数据都是primary给它准备的,需要确保其启动时可见。
				从实现看hart\_start并没有包含内存屏障,所以建议封装一下该函数,发出sbi调用之前强制加上fence(iorw,iorw)。这样比较保险。
			\end{block}
			\item 通过loader启动Linux环境:展开文件系统,加载并启动第一个程序,默认shell
			\item 通过loader启动Zircon环境:这个对baremetal riscv平台没有实现(boot\_library宏目前不支持)
		\end{enumerate}
	\end{frame}

	\begin{frame}
		\frametitle{6 memory管理}
		\begin{enumerate}
			\item 初始化全局的HEAP,类型BuddyAllocator,rust用global\_allocator进行标记:\\
			BuddyAllocator的实现与Linux kernel的page\_alloc中定义的Buddy系统基本一致。
			模板参数N指定Order桶的数量(27个),桶采用单链表结构,此外用USIZE实现位图标记各级桶的空闲/占用情况。
			\item 填充内存给全局HEAP
			\item 在内核数据段中预先保留一段内存区域,调用transfer方法转给全局的HEAP,内部是调用BuddyAllocator的回收方法deallocate来实现的。
			\item 回收方法deallocate:\\
			与linux kernel实现类似:从待回收页块所在的Order开始,逐级向上检测它的伙伴块是否已经是空闲的。
			如果空闲,合并后向上级Order迭代此过程;否则,直接插入到当前Order,完成回收过程。
			实现特色是,顶级Order的块因为不合并,加入寡头链特殊处理了。
			\item 分配方法:与linux kernel实现类似
		\end{enumerate}
	\end{frame}

	\begin{frame}
		\frametitle{6 memory管理}
		\begin{enumerate}
			\item KernelMemInfo【zCore/src/platform/riscv/consts.rs】\\
			用于提供内核虚存空间到物理地址空间的偏移值。
			\begin{block}{}
				初始化时用了固定数值0xffff\_ffc0\_8020\_0000作为vaddr\_base,我理解可以分解成三个值:\\
				(1)内核在虚拟内存空间中的开始地址 ≈ 0xffff\_ffc0\_0000\_0000,这个值应该参数化。目前这个值适用于SV39,将来采用其它mmu模式,它也需要相应改变。\\
				(2) 0x8000\_0000这个值应该是利用RISCV平台的一个惯例,RAM的物理空间地址通常就是它。
				这样计算va\_pa\_offset时可以消掉这个8,达成一个1G对齐的偏移,简化启动阶段页表初始化。但是这个惯例不是强制规范,依赖它可能导致一些意外。\\
				(3) 0x20\_0000即RiscV64系统需要把Kernel image加载到RAM开始地址后面2M处,这个应该是SBI规范所规定的。
				既然上面第2条不可靠,建议参照Linux kernel 和 zircon的通常做法,直接把第1条的那个值参数化后,作为vaddr\_base。
				此外,初始化页表时应该以PMD的尺寸为粒度,PMD size是2M,符合第3条,可以保证对齐要求。
			\end{block}
		\end{enumerate}
	\end{frame}

	\begin{frame}
		\frametitle{7 primary\_init\_early:解析设备树文件DTB,设置系统参数}
		\begin{enumerate}
			\item 系统命令行参数,即cmdline,来自dtb.chosen.bootargs
			\item CPU的时钟主频,后面设置定时器等时间相关操作时能用到
			\item 如果initrd存在,它在物理空间中占据的范围
			\item 可用的内存(基本上就是RAM)在物理空间中的范围。\\
			通常都是从0x8000\_0000开始。
			有可能存在多个不连续区间,所以需要用vec一类的数组或链来管理
		\end{enumerate}
	\end{frame}

	\begin{frame}
		\frametitle{8 Primary正式初始化:HAL::primary\_init}
		\begin{enumerate}
			\item 虚拟内存初始化vm::init:\\
			映射kernel image占据的物理空间范围到虚拟内存空间。\\
			(1)解析kernel image在物理空间中各段的位置(code, data, etc.)\\
			(2)对各段执行线性偏移映射,各段中间可能有空洞,最后应该需要特殊处理以提高安全性。
			\item 内核驱动的初始化
			\item 遍历DTB发现所有的设备,把如下类型的设备加入管理列表\\
			Uart,PCI,intc,Display和Net
			\item 根中断控制器intc的初始化:\\
			根中断控制器intc是Hart内置的中断控制器,接入三类中断:
			(1)时钟中断:该中断是通过SBI从Machine模式转过来的,最主要的作用是协助调度器实现时间片资源管理。
			(2)软件中断:也是通过SBI从Machine模式转过来的,用于实现IPI。
			(3)外部中断:经由平台级中断控制器Plic的仲裁分配。
			\item 为时钟中断和软件中断注册handler:\\
			对于时钟中断,每次发生中断时,都要通过SBI重置定时器,使之在固定的周期后再次触发。
			\item 允许时钟中断和软件中断:就是在寄存器SIE上置相应位。
		\end{enumerate}
	\end{frame}

	\begin{frame}
		\frametitle{9 建立linux环境,启动第一个进程}
		\begin{enumerate}
			\item 创建Job、Process、Thread三级执行体
			\item 基于LinuxELF格式解析器,解析第一个程序(默认是busybox?sh),返回程序入口地址entry和用户栈指针位置
			\item 启动Thread从entry开始执行\\
			(1)zCore特点:用户进程/线程基于async机制实现,所以这个entry被封装成一个future
			(2)启动用户线程就变成了把上述future加入到执行器executor中,具体的执行调度在PreemptScheduler中实现
			\item 为时钟中断和软件中断注册handler:\\
			对于时钟中断,每次发生中断时,都要通过SBI重置定时器,使之在固定的周期后再次触发。
			\item 允许时钟中断和软件中断:就是在寄存器SIE上置相应位。
		\end{enumerate}
	\end{frame}

	\begin{frame}
		\frametitle{10 PreemptScheduler}
		\begin{enumerate}
			\item 核心数据结构RutimeExecutor:\\
			每个Hart对应一个runtime(RutimeExecutor类型),作用相当于linux kernel的idle kthread。
			\begin{block}{}
			zCore目前是根据固定数量来初始化runtime。
			前面已经根据dtb信息获知实际Hart的数量,并建立了对应的hart数组,
			所以这里可以根据实际数量来初始化runtime并作为资源纳入对应的hart进行管理,不必定义全局变量。
			\end{block}
			\item spawn生成一个由future包装的线程并加入待调度列表
			\item 在关中断的条件下运行
			\item 找到当前Hart的runtime,把该future加入到runtime的任务列表中
			\item wait\_until\_idle:\\
			开始执行调度,strong\_executor强执行器具备高优先级,用于执行当前future,但是如果没能在一个时间片内完成,就会被时钟中断打断运行,降级为weak\_executor;
			weak\_executor只会在没有strong\_executor时才能得到机会执行。通过strong/weak executor以及时钟中断(调用handle\_timeout)之间的协作,实现可抢占的调度。
		\end{enumerate}
	\end{frame}

	\begin{frame}
		\frametitle{10 PreemptScheduler}
		\begin{enumerate}
			\item wait\_for\_interrupt:\\
			\begin{block}{}
				在开中断前加内存屏障fense(iorw, iorw)
			\end{block}
			\item 保存当前中断状态,开中断
			\begin{block}{}
				当前实现是分别使用读sstatus状态和set\_sie两个操作来完成。
				实际上set\_sie封装的指令CSRRS,是一个设置新值同时返回旧值的原子操作,crate riscv实现时忽略了返回值。我们可以修改一下其实现。
			\end{block}
			\item 进入等待状态,直到有中断发生时被唤醒。
			\item 恢复等待前的中断状态
			\item switch:\\
			执行器executor之间通过switch实现切换,从这个角度,执行器就等同于Linux kernel的一个kthread,切换的机制相同。
			\begin{block}{}
				Switch入口处需要加内存屏障fence(iorw, iorw)。
				即在当前执行体被切换出去之前,自己清算一下内存操作,让其它执行体可见。
			\end{block}
		\end{enumerate}
	\end{frame}

	\begin{frame}
		\frametitle{杂项}
		\begin{enumerate}
			\item Rust的内存屏障core::sync::atomic::fence\\
			\begin{block}{}
				这个函数不能用,通过反汇编发现,即使传入SeqCst参数,也只能产生fence(rw,rw)的指令,而我们通常需要的是fence(iorw,iorw)。
				所以需要用asm自己封装。
			\end{block}
			\item Panic\\
			\begin{block}{}
				我们通常会用qemu来调试,遇到panic时qemu不会自己退出,还要输入ctl+A+X,写测试脚本时不方便。
				建议在panic中把spin\_loop无限循环改为关机(sbi.legacy.shutdown或者srst.system\_reset with RESET\_TYPE\_SHUTDOWN),这样panic就可以让qemu退出。
			\end{block}
			\item TicketMutex[kernel-sync/ticket.rs]
			\begin{block}{}
				对self.next\_ticket的操作缺少内存屏障,既然lock中fetch\_add和is\_locked中的load都是Ordering::Relaxed,
				那么某些cpu在调用is\_locked时,可能会取到next\_ticket的陈旧值。
			\end{block}
		\end{enumerate}
	\end{frame}

	\begin{frame}
		\frametitle{关于GP寄存器}
		没有发现zCore内核里面用GP寄存器relaxing机制,建议可以参照linux kernel用一下这个riscv特色。
		linker.ld中.data和.bss段之间定义一个\_\_global\_pointer\$
		然后入口处启用paging前后各初始化一次gp寄存器
		此外,外部crate trapframe在异常入口位置也要加一下,就在保存S态系统寄存器之后的位置
		\begin{quote}
			\# save sp, sstatus, sepc\\
			… …\\
			STORE\_SP t2, 33         \# save sepc\\
			\vspace {8pt} 
			/* Load the global pointer */\\
			.option push\\
			.option norelax\\
			la gp, \_\_global\_pointer\$\\
			.option pop\\
			\vspace {8pt} 
			andi t1, t1, 1 << 8     \# sstatus.SPP = 1\\
			beqz t1, end\_trap\_from\_user\\
			end\_trap\_from\_kernel:
		\end{quote}
	\end{frame}

	%!!!!!!!!!!!!!!!!!!!!!!!!!!!!!!!!!!!!!!!!!!!!!!!!!!!!!!!!!!!!!!!
	%	Demo
	%!!!!!!!!!!!!!!!!!!!!!!!!!!!!!!!!!!!!!!!!!!!!!!!!!!!!!!!!!!!!!!!

	\section{Text Examples}

	\subsection{Paragraphs and Lists}

	\begin{frame}
	\frametitle{Paragraphs of Text}

	Sed iaculis \alert{dapibus gravida}. Morbi sed tortor erat, nec interdum arcu. Sed id lorem lectus. Quisque viverra augue id sem ornare non aliquam nibh tristique. Aenean in ligula nisl. Nulla sed tellus ipsum. Donec vestibulum ligula non lorem vulputate fermentum accumsan neque mollis.

	\bigskip % Vertical whitespace

	\begin{quote}
		Sed diam enim, sagittis nec condimentum sit amet, ullamcorper sit amet libero. Aliquam vel dui orci, a porta odio.\\
		--- Someone, somewhere\ldots
	\end{quote}

	\bigskip % Vertical whitespace

	Nullam id suscipit ipsum. Aenean lobortis commodo sem, ut commodo leo gravida vitae. Pellentesque vehicula ante iaculis arcu pretium rutrum eget sit amet purus. Integer ornare nulla quis neque ultrices lobortis.
	\end{frame}

	\subsection{Blocks}
	
	\begin{frame}
		\frametitle{Blocks of Highlighted Text}
		
		\begin{block}{Block Title}
			Lorem ipsum dolor sit amet, consectetur adipiscing elit. Integer lectus nisl, ultricies in feugiat rutrum, porttitor sit amet augue.
		\end{block}
		
		\begin{exampleblock}{Example Block Title}
			Aliquam ut tortor mauris. Sed volutpat ante purus, quis accumsan.
		\end{exampleblock}
		
		\begin{alertblock}{Alert Block Title}
			Pellentesque sed tellus purus. Class aptent taciti sociosqu ad litora torquent per conubia nostra, per inceptos himenaeos.
		\end{alertblock}
		
		\begin{block}{} % Block without title
			Suspendisse tincidunt sagittis gravida. Curabitur condimentum, enim sed venenatis rutrum, ipsum neque consectetur orci.
		\end{block}
	\end{frame}

	\subsection{Columns}
	
	\begin{frame}
		\frametitle{Multiple Columns}
		\framesubtitle{Subtitle} % Optional subtitle
		
		\begin{columns}[c] % The "c" option specifies centered vertical alignment while the "t" option is used for top vertical alignment
			\begin{column}{0.45\textwidth} % Left column width
				\textbf{Heading}
				\begin{enumerate}
					\item Statement
					\item Explanation
					\item Example
				\end{enumerate}
			\end{column}
			\begin{column}{0.5\textwidth} % Right column width
				Lorem ipsum dolor sit amet, consectetur adipiscing elit. Integer lectus nisl, ultricies in feugiat rutrum, porttitor sit amet augue. Aliquam ut tortor mauris. Sed volutpat ante purus, quis accumsan dolor.
			\end{column}
		\end{columns}
	\end{frame}

	\section{Table and Figure Examples}
	
	\subsection{Table}
	
	\begin{frame}
		\frametitle{Table}
		\framesubtitle{Subtitle} % Optional subtitle
		
		\begin{table}
			\begin{tabular}{l l l}
				\toprule
				\textbf{Treatments} & \textbf{Response 1} & \textbf{Response 2}\\
				\midrule
				Treatment 1 & 0.0003262 & 0.562 \\
				Treatment 2 & 0.0015681 & 0.910 \\
				Treatment 3 & 0.0009271 & 0.296 \\
				\bottomrule
			\end{tabular}
			\caption{Table caption}
		\end{table}
	\end{frame}

	\subsection{Figure}
	
	\begin{frame}
		\frametitle{Figure}
		
		\begin{figure}
			\includegraphics[width=0.8\linewidth]{creodocs_logo.pdf}
			\caption{Creodocs logo.}
		\end{figure}
	\end{frame}

	\section{Mathematics}
	
	\begin{frame}
		\frametitle{Definitions \& Examples}
		
		\begin{definition}
			A \alert{prime number} is a number that has exactly two divisors.
		\end{definition}
		
		\smallskip % Vertical whitespace
		
		\begin{example}
			\begin{itemize}
				\item 2 is prime (two divisors: 1 and 2).
				\item 3 is prime (two divisors: 1 and 3).
				\item 4 is not prime (\alert{three} divisors: 1, 2, and 4).
			\end{itemize}
		\end{example}
		
		\smallskip % Vertical whitespace
		
		You can also use the \texttt{theorem}, \texttt{lemma}, \texttt{proof} and \texttt{corollary} environments.
	\end{frame}

	\section{Referencing}
	
	\begin{frame}
		\frametitle{Citing References}
		
		An example of the \texttt{\textbackslash cite} command to cite within the presentation:
		
		\bigskip % Vertical whitespace
		
		This statement requires citation \cite{p1,p2}.
	\end{frame}
	
	%------------------------------------------------
	
	\begin{frame} % Use [allowframebreaks] to allow automatic splitting across slides if the content is too long
		\frametitle{References}
		
		\begin{thebibliography}{99} % Beamer does not support BibTeX so references must be inserted manually as below, you may need to use multiple columns and/or reduce the font size further if you have many references
			\footnotesize % Reduce the font size in the bibliography
			
			\bibitem[Smith, 2022]{p1}
			John Smith (2022)
			\newblock Publication title
			\newblock \emph{Journal Name} 12(3), 45 -- 678.
			
			\bibitem[Kennedy, 2023]{p2}
			Annabelle Kennedy (2023)
			\newblock Publication title
			\newblock \emph{Journal Name} 12(3), 45 -- 678.
		\end{thebibliography}
	\end{frame}

	\begin{frame}
		\frametitle{Acknowledgements}
		
		\begin{columns}[t] % The "c" option specifies centered vertical alignment while the "t" option is used for top vertical alignment
			\begin{column}{0.45\textwidth} % Left column width
				\textbf{Smith Lab}
				\begin{itemize}
					\item Alice Smith
					\item Devon Brown
				\end{itemize}
				\textbf{Cook Lab}
				\begin{itemize}
					\item Margaret
					\item Jennifer
					\item Yuan
				\end{itemize}
			\end{column}		
			\begin{column}{0.5\textwidth} % Right column width
				\textbf{Funding}
				\begin{itemize}
					\item British Royal Navy
					\item Norwegian Government
				\end{itemize}
			\end{column}
		\end{columns}
	\end{frame}

	\begin{frame}[plain] % The optional argument 'plain' hides the headline and footline
		\begin{center}
			{\Huge The End}
			
			\bigskip\bigskip % Vertical whitespace
			
			{\LARGE Questions? Comments?}
		\end{center}
	\end{frame}

\end{document}