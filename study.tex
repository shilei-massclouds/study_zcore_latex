\documentclass[
11pt, % Set the default font size, options include: 8pt, 9pt, 10pt, 11pt, 12pt, 14pt, 17pt, 20pt
%t, % Uncomment to vertically align all slide content to the top of the slide, rather than the default centered
%aspectratio=169, % Uncomment to set the aspect ratio to a 16:9 ratio which matches the aspect ratio of 1080p and 4K screens and projectors
]{beamer}

\graphicspath{{Images/}{./}}

\usepackage{booktabs}
\usepackage{ctex}

\usetheme{Madrid}

\usefonttheme{default}

\usepackage{palatino}
\usepackage[default]{opensans}

\useinnertheme{circles}

%----------------------------------------------------------------------------------------
%	PRESENTATION INFORMATION
%----------------------------------------------------------------------------------------
\title{参与研究zCore的初步想法}
\institute[Massclouds]{乾云科技}
%----------------------------------------------------------------------------------------

\begin{document}

	%----------------------------------------------------------------------------------------
	%	TITLE SLIDE
	%----------------------------------------------------------------------------------------
	\begin{frame}
		\titlepage
	\end{frame}

	%----------------------------------------------------------------------------------------
	%	OBJECTIVE
	%----------------------------------------------------------------------------------------
	\begin{frame}
		\frametitle{目标}
		从zCore起步,参与安全操作系统的研究工作:
		\begin{itemize}
			\item 发展一个面向AIoT领域的安全操作系统,以RISCV为主要目标体系结构,向上提供与Zircon保持同步的系统调用接口 以及 与Linux kernel保持兼容的系统调用接口。
			\item 以Rust-lang为主要的实现语言,架构上参照Zircon的整体架构设计,具体细节参照Linux kernel(arch/riscv)。
			\item 通过开源社区的方式组织研究和开发,建立各层面开发者参与协作的通道,建立应用层面使用者的反馈通道。
		\end{itemize}
	\end{frame}

	%----------------------------------------------------------------------------------------
	%	TABLE OF CONTENTS SLIDE
	%----------------------------------------------------------------------------------------
	\begin{frame}
	\frametitle{纲要}
		\tableofcontents
	\end{frame}

	%----------------------------------------------------------------------------------------
	%	PRESENTATION BODY SLIDES
	%----------------------------------------------------------------------------------------
	\section{Text Examples}

	\subsection{Paragraphs and Lists}

	\begin{frame}
	\frametitle{Paragraphs of Text}

	Sed iaculis \alert{dapibus gravida}. Morbi sed tortor erat, nec interdum arcu. Sed id lorem lectus. Quisque viverra augue id sem ornare non aliquam nibh tristique. Aenean in ligula nisl. Nulla sed tellus ipsum. Donec vestibulum ligula non lorem vulputate fermentum accumsan neque mollis.

	\bigskip % Vertical whitespace

	\begin{quote}
		Sed diam enim, sagittis nec condimentum sit amet, ullamcorper sit amet libero. Aliquam vel dui orci, a porta odio.\\
		--- Someone, somewhere\ldots
	\end{quote}

	\bigskip % Vertical whitespace

	Nullam id suscipit ipsum. Aenean lobortis commodo sem, ut commodo leo gravida vitae. Pellentesque vehicula ante iaculis arcu pretium rutrum eget sit amet purus. Integer ornare nulla quis neque ultrices lobortis.
	\end{frame}

	\subsection{Blocks}
	
	\begin{frame}
		\frametitle{Blocks of Highlighted Text}
		
		\begin{block}{Block Title}
			Lorem ipsum dolor sit amet, consectetur adipiscing elit. Integer lectus nisl, ultricies in feugiat rutrum, porttitor sit amet augue.
		\end{block}
		
		\begin{exampleblock}{Example Block Title}
			Aliquam ut tortor mauris. Sed volutpat ante purus, quis accumsan.
		\end{exampleblock}
		
		\begin{alertblock}{Alert Block Title}
			Pellentesque sed tellus purus. Class aptent taciti sociosqu ad litora torquent per conubia nostra, per inceptos himenaeos.
		\end{alertblock}
		
		\begin{block}{} % Block without title
			Suspendisse tincidunt sagittis gravida. Curabitur condimentum, enim sed venenatis rutrum, ipsum neque consectetur orci.
		\end{block}
	\end{frame}

	\subsection{Columns}
	
	\begin{frame}
		\frametitle{Multiple Columns}
		\framesubtitle{Subtitle} % Optional subtitle
		
		\begin{columns}[c] % The "c" option specifies centered vertical alignment while the "t" option is used for top vertical alignment
			\begin{column}{0.45\textwidth} % Left column width
				\textbf{Heading}
				\begin{enumerate}
					\item Statement
					\item Explanation
					\item Example
				\end{enumerate}
			\end{column}
			\begin{column}{0.5\textwidth} % Right column width
				Lorem ipsum dolor sit amet, consectetur adipiscing elit. Integer lectus nisl, ultricies in feugiat rutrum, porttitor sit amet augue. Aliquam ut tortor mauris. Sed volutpat ante purus, quis accumsan dolor.
			\end{column}
		\end{columns}
	\end{frame}

	\section{Table and Figure Examples}
	
	\subsection{Table}
	
	\begin{frame}
		\frametitle{Table}
		\framesubtitle{Subtitle} % Optional subtitle
		
		\begin{table}
			\begin{tabular}{l l l}
				\toprule
				\textbf{Treatments} & \textbf{Response 1} & \textbf{Response 2}\\
				\midrule
				Treatment 1 & 0.0003262 & 0.562 \\
				Treatment 2 & 0.0015681 & 0.910 \\
				Treatment 3 & 0.0009271 & 0.296 \\
				\bottomrule
			\end{tabular}
			\caption{Table caption}
		\end{table}
	\end{frame}

	\subsection{Figure}
	
	\begin{frame}
		\frametitle{Figure}
		
		\begin{figure}
			\includegraphics[width=0.8\linewidth]{creodocs_logo.pdf}
			\caption{Creodocs logo.}
		\end{figure}
	\end{frame}

	\section{Mathematics}
	
	\begin{frame}
		\frametitle{Definitions \& Examples}
		
		\begin{definition}
			A \alert{prime number} is a number that has exactly two divisors.
		\end{definition}
		
		\smallskip % Vertical whitespace
		
		\begin{example}
			\begin{itemize}
				\item 2 is prime (two divisors: 1 and 2).
				\item 3 is prime (two divisors: 1 and 3).
				\item 4 is not prime (\alert{three} divisors: 1, 2, and 4).
			\end{itemize}
		\end{example}
		
		\smallskip % Vertical whitespace
		
		You can also use the \texttt{theorem}, \texttt{lemma}, \texttt{proof} and \texttt{corollary} environments.
	\end{frame}

	\section{Referencing}
	
	\begin{frame}
		\frametitle{Citing References}
		
		An example of the \texttt{\textbackslash cite} command to cite within the presentation:
		
		\bigskip % Vertical whitespace
		
		This statement requires citation \cite{p1,p2}.
	\end{frame}
	
	%------------------------------------------------
	
	\begin{frame} % Use [allowframebreaks] to allow automatic splitting across slides if the content is too long
		\frametitle{References}
		
		\begin{thebibliography}{99} % Beamer does not support BibTeX so references must be inserted manually as below, you may need to use multiple columns and/or reduce the font size further if you have many references
			\footnotesize % Reduce the font size in the bibliography
			
			\bibitem[Smith, 2022]{p1}
			John Smith (2022)
			\newblock Publication title
			\newblock \emph{Journal Name} 12(3), 45 -- 678.
			
			\bibitem[Kennedy, 2023]{p2}
			Annabelle Kennedy (2023)
			\newblock Publication title
			\newblock \emph{Journal Name} 12(3), 45 -- 678.
		\end{thebibliography}
	\end{frame}

	\begin{frame}
		\frametitle{Acknowledgements}
		
		\begin{columns}[t] % The "c" option specifies centered vertical alignment while the "t" option is used for top vertical alignment
			\begin{column}{0.45\textwidth} % Left column width
				\textbf{Smith Lab}
				\begin{itemize}
					\item Alice Smith
					\item Devon Brown
				\end{itemize}
				\textbf{Cook Lab}
				\begin{itemize}
					\item Margaret
					\item Jennifer
					\item Yuan
				\end{itemize}
			\end{column}		
			\begin{column}{0.5\textwidth} % Right column width
				\textbf{Funding}
				\begin{itemize}
					\item British Royal Navy
					\item Norwegian Government
				\end{itemize}
			\end{column}
		\end{columns}
	\end{frame}

	\begin{frame}[plain] % The optional argument 'plain' hides the headline and footline
		\begin{center}
			{\Huge The End}
			
			\bigskip\bigskip % Vertical whitespace
			
			{\LARGE Questions? Comments?}
		\end{center}
	\end{frame}

\end{document}