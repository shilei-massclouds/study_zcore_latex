\documentclass[
8pt, % Set the default font size, options include: 8pt, 9pt, 10pt, 11pt, 12pt, 14pt, 17pt, 20pt
%t, % Uncomment to vertically align all slide content to the top of the slide, rather than the default centered
%aspectratio=169, % Uncomment to set the aspect ratio to a 16:9 ratio which matches the aspect ratio of 1080p and 4K screens and projectors
]{beamer}

\graphicspath{{Images/}{./}}

\usepackage{booktabs}
\usepackage{ctex}

\usetheme{Madrid}

\usefonttheme{default}

\usepackage{palatino}
\usepackage[default]{opensans}

\useinnertheme{circles}

%----------------------------------------------------------------------------------------
%	PRESENTATION INFORMATION
%----------------------------------------------------------------------------------------
\title{参与研究安全操作系统的初步想法}
\institute[Massclouds]{\large 乾云科技}
%----------------------------------------------------------------------------------------

\begin{document}

	%----------------------------------------------------------------------------------------
	%	TITLE SLIDE
	%----------------------------------------------------------------------------------------
	\begin{frame}
		\titlepage
	\end{frame}

	%----------------------------------------------------------------------------------------
	%	OBJECTIVE
	%----------------------------------------------------------------------------------------
	\begin{frame}
		\frametitle{目标}
		{\large 从zCore起步,参与安全操作系统的研究工作:}
		\begin{itemize}
			\item {\large 发展一个面向AIoT领域的安全操作系统,以RISCV为主要目标体系结构,向上提供与Zircon保持同步的系统调用接口 以及 与Linux kernel保持兼容的系统调用接口。}
			\item {\large 以Rust-lang为主要的实现语言,架构上参照Zircon的整体架构设计,具体细节参照Linux kernel(arch/riscv)。}
			\item {\large 通过开源社区的方式组织研究和开发,建立各层面开发者参与协作的通道,建立应用层面使用者的反馈通道。}
		\end{itemize}
		{\large 下面就目前进行的准备工作和对将来工作的设想做一个汇报:}
	\end{frame}

	%----------------------------------------------------------------------------------------
	%	TABLE OF CONTENTS SLIDE
	%----------------------------------------------------------------------------------------
	\begin{frame}
		\frametitle{纲要}
		\tableofcontents
	\end{frame}

	%----------------------------------------------------------------------------------------
	%	PRESENTATION BODY SLIDES
	%----------------------------------------------------------------------------------------
	\section{zCore学习过程和疑问}

	\begin{frame}
		\begin{center}
			{\LARGE 对zCore的学习过程及疑问\\}
			\bigskip\bigskip
			{\large 仅针对BareMetal RISCV平台 - 9月15日版本}
		\end{center}
	\end{frame}

	\subsection{引导阶段}
	
	\begin{frame}
		\frametitle{1 入口\_start}
		\begin{enumerate}
			\item 通过select\_stack为primary hart查找和配置内核栈,为函数调用作准备
			\begin{block}{}
			建议这里不必按照固定的最大值(MAX\_HART\_NUM)为所有的harts预分配页。此时还不知道Harts的总数。可以只为primary hart在bss中预分配栈,等primary hart初始化过程中确定harts数量之后,再按需分配。启动secondary harts可以改一下,把它们的启动时机推后,具体想法后面会提到。
			\end{block}
			\item 进入primary hart的启动函数primary\_rust\_main
			\item 清零BSS段
			\item 初始化页表,为切换到虚拟空间作准备(见2)
			\item 检查设备树dtb有效性
			\item 启动secondary harts(见3)
			\item 进入下阶段primary初始化
		\end{enumerate}
	\end{frame}

	\begin{frame}
		\frametitle{2 启动阶段页表设置boot\_page\_table}
		\begin{enumerate}
			\item 设置物理地址与虚拟地址相等的跳板页
			\item 从虚存开始地址映射物理空间的前128G
			\item 基于SV39模式启用mmu
			\begin{block}{}
				目前zCore是固定采用sv39的模式。可以借鉴Linux kernel的方式,通过写后读轮流尝试sv57,sv48来自动检测所在硬件平台支持的最大模式。
			\end{block}
			\item 调整栈寄存器sp和返回地址寄存器ra,完成从物理空间到虚拟空间转换
			\item 设置状态寄存器,从此允许kernel访问User页面
			\begin{block}{}
				建议采取保守策略,默认情况下始终不开,即默认不允许内核直接访问用户页面,仅在必须执行kernel/user之间内存拷贝或设置的时候,再临时打开,操作完后随即关闭。
			\end{block}
		\end{enumerate}
	\end{frame}

	\begin{frame}
		\frametitle{3 启动Secondary Harts}
		\begin{enumerate}
			\item 检查sbi是否支持HSM扩展,不支持则退化为单核系统
			\begin{block}{}
				HSM是OpenSBIv0.7之后才支持,可以考虑参考Linux kernel for riscv的处理方式:\\
				(1)这里不检查HSM的支持。\\
				(2)把那个Atomic全局变量STARTED提前到\_start入口位置,用它保证只有一个Hart能通过,其它在STARTED变量spin等待。那个唯一能通过的就是Primary Hart。\\
				(3)Primary Hart启动各secondary harts时,如支持HSM,就如当前的办法用SBI唤醒;不支持的,打开STARTED的封锁。
			\end{block}
			\item 遍历设备树文件dtb,发现primary之外所有的hart,通过SBI启动它们,执行secondary的启动流程(见4)。
			\begin{block}{}
				感觉这里就启动secondary harts有点早了,一是这里需要单独处理dtb,其它几处还有单独处理dtb的地方,代码有点分散;二是启动了后面也没有真正运行,还是要等primary的信号。
				建议由Primary Hart在一处统一处理dtb,生成各种管理结构,包括Harts对象数组,然后为所有secondary harts准备好栈,再kick它们。
			\end{block}
		\end{enumerate}
	\end{frame}

	\begin{frame}
		\frametitle{4 Secondary Hart的启动流程}
		\begin{enumerate}
			\item 设置栈:zCore直接用hartid这个硬件ID作为CPU的编号,用它来对应包括系统栈在内的各种CPU专属资源。
			\begin{block}{}
				直接利用hartid作为CPU编号有点问题。
				按照Riscv规范【The RISC-V Instruction Set Manual - Volume II: Privileged Architecture】3.1.5节:
				至少有一个HART的hartid是零,但并不要求所有hartid是连续的。实现策略由厂商自行决定。
				由于不能保证hartid是从零开始的连续数值区间,所以当索引不合适。建议按照通常做法,维护一个CPU数组,CPU在数组的位置即逻辑ID作为CPU资源的索引。
			\end{block}
			\item zCore占用TP寄存器来维持hartid。
			\begin{block}{}
				对每个hart来说,0号系统寄存器就是harid只读寄存器,不必再占用tp寄存器。tp可以用来保持cpu逻辑ID或者直接保存对应percpu的引用指针。
			\end{block}
			\item 通过循环实现HART对应内核栈的定位。
			\begin{block}{}
				应该用索引+移位实现就可以吧?
			\end{block}
		\end{enumerate}
	\end{frame}

	\begin{frame}
		\frametitle{4 Secondary Hart的启动流程}
		\begin{enumerate}
			\item 启用分页
			\item 逐级进入secondary主函数secondary\_main,最后通过HAL的方法完成初始化
			\item 执行HAL初始化:通过kernel\_hal::secondary\_init
			\item 初始化本Hart的根中断控制器intc
			\item 初始化平台级中断控制器plic
			\begin{block}{}
				把plic放到secondary hart的初始化过程中不合适,如前所述,如果SBI不支持HSM,就不仅是退化到单核的问题,连设备中断也没有了。
				这个plic是平台级的,只需要初始化一次,建议还是由primary hart初始化为妥。
			\end{block}
		\end{enumerate}
	\end{frame}

	\begin{frame}
		\frametitle{5 primary\_main主函数}
		\begin{block}{}
			建议用一个Context包装原本的全局变量,在此处定义后(是局部变量),作为参数一直传递下去,直至准备启动secondary harts
			在准备启动secondary harts前,释放掉不再需要的,其它的用ARC和Mutex封装后clone,传递给那些secondary harts
		\end{block}
		\begin{enumerate}
			\item 日志初始化
			\item 内存初始化
			\item HAL层对primary hart的早期初始化
			\item 把从dtb中发现的可用内存范围加入到系统堆中
			\item HAL层对primary hart的正式初始化
			\item 唤醒所有处于等待状态的secondary harts
			\begin{block}{}
				primary hart在唤醒secondary hart之前应该先使用内存屏障,毕竟secondary将要使用的内存数据都是primary给它准备的,需要确保其启动时可见。
				从实现看hart\_start并没有包含内存屏障,所以建议封装一下该函数,发出sbi调用之前强制加上fence(iorw,iorw)。这样比较保险。
			\end{block}
			\item 通过loader启动Linux环境:展开文件系统,加载并启动第一个程序,默认shell
			\item 通过loader启动Zircon环境:这个对baremetal riscv平台没有实现(boot\_library宏目前不支持)
		\end{enumerate}
	\end{frame}

	\begin{frame}
		\frametitle{6 memory管理}
		\begin{enumerate}
			\item 初始化全局的HEAP,类型BuddyAllocator,rust用global\_allocator进行标记:\\
			BuddyAllocator的实现与Linux kernel的page\_alloc中定义的Buddy系统基本一致。
			模板参数N指定Order桶的数量(27个),桶采用单链表结构,此外用USIZE实现位图标记各级桶的空闲/占用情况。
			\item 填充内存给全局HEAP
			\item 在内核数据段中预先保留一段内存区域,调用transfer方法转给全局的HEAP,内部是调用BuddyAllocator的回收方法deallocate来实现的。
			\item 回收方法deallocate:\\
			与linux kernel实现类似:从待回收页块所在的Order开始,逐级向上检测它的伙伴块是否已经是空闲的。
			如果空闲,合并后向上级Order迭代此过程;否则,直接插入到当前Order,完成回收过程。
			实现特色是,顶级Order的块因为不合并,加入寡头链特殊处理了。
			\item 分配方法:与linux kernel实现类似
		\end{enumerate}
	\end{frame}

	\begin{frame}
		\frametitle{6 memory管理}
		\begin{enumerate}
			\item KernelMemInfo【zCore/src/platform/riscv/consts.rs】\\
			用于提供内核虚存空间到物理地址空间的偏移值。
			\begin{block}{}
				初始化时用了固定数值0xffff\_ffc0\_8020\_0000作为vaddr\_base,我理解可以分解成三个值:\\
				(1)内核在虚拟内存空间中的开始地址 ≈ 0xffff\_ffc0\_0000\_0000,这个值应该参数化。目前这个值适用于SV39,将来采用其它mmu模式,它也需要相应改变。\\
				(2) 0x8000\_0000这个值应该是利用RISCV平台的一个惯例,RAM的物理空间地址通常就是它。
				这样计算va\_pa\_offset时可以消掉这个8,达成一个1G对齐的偏移,简化启动阶段页表初始化。但是这个惯例不是强制规范,依赖它可能导致一些意外。\\
				(3) 0x20\_0000即RiscV64系统需要把Kernel image加载到RAM开始地址后面2M处,这个应该是SBI规范所规定的。
				既然上面第2条不可靠,建议参照Linux kernel 和 zircon的通常做法,直接把第1条的那个值参数化后,作为vaddr\_base。
				此外,初始化页表时应该以PMD的尺寸为粒度,PMD size是2M,符合第3条,可以保证对齐要求。
			\end{block}
		\end{enumerate}
	\end{frame}

	\begin{frame}
		\frametitle{7 primary\_init\_early:解析设备树文件DTB,设置系统参数}
		\begin{enumerate}
			\item 系统命令行参数,即cmdline,来自dtb.chosen.bootargs
			\item CPU的时钟主频,后面设置定时器等时间相关操作时能用到
			\item 如果initrd存在,它在物理空间中占据的范围
			\item 可用的内存(基本上就是RAM)在物理空间中的范围。\\
			通常都是从0x8000\_0000开始。
			有可能存在多个不连续区间,所以需要用vec一类的数组或链来管理
		\end{enumerate}
	\end{frame}

	\begin{frame}
		\frametitle{8 Primary正式初始化:HAL::primary\_init}
		\begin{enumerate}
			\item 虚拟内存初始化vm::init:\\
			映射kernel image占据的物理空间范围到虚拟内存空间。\\
			(1)解析kernel image在物理空间中各段的位置(code, data, etc.)\\
			(2)对各段执行线性偏移映射,各段中间可能有空洞,最后应该需要特殊处理以提高安全性。
			\item 内核驱动的初始化
			\item 遍历DTB发现所有的设备,把如下类型的设备加入管理列表\\
			Uart,PCI,intc,Display和Net
			\item 根中断控制器intc的初始化:\\
			根中断控制器intc是Hart内置的中断控制器,接入三类中断:
			(1)时钟中断:该中断是通过SBI从Machine模式转过来的,最主要的作用是协助调度器实现时间片资源管理。
			(2)软件中断:也是通过SBI从Machine模式转过来的,用于实现IPI。
			(3)外部中断:经由平台级中断控制器Plic的仲裁分配。
			\item 为时钟中断和软件中断注册handler:\\
			对于时钟中断,每次发生中断时,都要通过SBI重置定时器,使之在固定的周期后再次触发。
			\item 允许时钟中断和软件中断:就是在寄存器SIE上置相应位。
		\end{enumerate}
	\end{frame}

	\begin{frame}
		\frametitle{9 建立linux环境,启动第一个进程}
		\begin{enumerate}
			\item 创建Job、Process、Thread三级执行体
			\item 基于LinuxELF格式解析器,解析第一个程序(默认是busybox?sh),返回程序入口地址entry和用户栈指针位置
			\item 启动Thread从entry开始执行\\
			(1)zCore特点:用户进程/线程基于async机制实现,所以这个entry被封装成一个future
			(2)启动用户线程就变成了把上述future加入到执行器executor中,具体的执行调度在PreemptScheduler中实现
			\item 为时钟中断和软件中断注册handler:\\
			对于时钟中断,每次发生中断时,都要通过SBI重置定时器,使之在固定的周期后再次触发。
			\item 允许时钟中断和软件中断:就是在寄存器SIE上置相应位。
		\end{enumerate}
	\end{frame}

	\begin{frame}
		\frametitle{10 PreemptScheduler}
		\begin{enumerate}
			\item 核心数据结构RutimeExecutor:\\
			每个Hart对应一个runtime(RutimeExecutor类型),作用相当于linux kernel的idle kthread。
			\begin{block}{}
			zCore目前是根据固定数量来初始化runtime。
			前面已经根据dtb信息获知实际Hart的数量,并建立了对应的hart数组,
			所以这里可以根据实际数量来初始化runtime并作为资源纳入对应的hart进行管理,不必定义全局变量。
			\end{block}
			\item spawn生成一个由future包装的线程并加入待调度列表
			\item 在关中断的条件下运行
			\item 找到当前Hart的runtime,把该future加入到runtime的任务列表中
			\item wait\_until\_idle:\\
			开始执行调度,strong\_executor强执行器具备高优先级,用于执行当前future,但是如果没能在一个时间片内完成,就会被时钟中断打断运行,降级为weak\_executor;
			weak\_executor只会在没有strong\_executor时才能得到机会执行。通过strong/weak executor以及时钟中断(调用handle\_timeout)之间的协作,实现可抢占的调度。
		\end{enumerate}
	\end{frame}

	\begin{frame}
		\frametitle{10 PreemptScheduler}
		\begin{enumerate}
			\item wait\_for\_interrupt:\\
			\begin{block}{}
				在开中断前加内存屏障fense(iorw, iorw)
			\end{block}
			\item 保存当前中断状态,开中断
			\begin{block}{}
				当前实现是分别使用读sstatus状态和set\_sie两个操作来完成。
				实际上set\_sie封装的指令CSRRS,是一个设置新值同时返回旧值的原子操作,crate riscv实现时忽略了返回值。我们可以修改一下其实现。
			\end{block}
			\item 进入等待状态,直到有中断发生时被唤醒。
			\item 恢复等待前的中断状态
			\item switch:\\
			执行器executor之间通过switch实现切换,从这个角度,执行器就等同于Linux kernel的一个kthread,切换的机制相同。
			\begin{block}{}
				Switch入口处需要加内存屏障fence(iorw, iorw)。
				即在当前执行体被切换出去之前,自己清算一下内存操作,让其它执行体可见。
			\end{block}
		\end{enumerate}
	\end{frame}

	\begin{frame}
		\frametitle{杂项}
		\begin{enumerate}
			\item Rust的内存屏障core::sync::atomic::fence\\
			\begin{block}{}
				这个函数不能用,通过反汇编发现,即使传入SeqCst参数,也只能产生fence(rw,rw)的指令,而我们通常需要的是fence(iorw,iorw)。
				所以需要用asm自己封装。
			\end{block}
			\item Panic\\
			\begin{block}{}
				我们通常会用qemu来调试,遇到panic时qemu不会自己退出,还要输入ctl+A+X,写测试脚本时不方便。
				建议在panic中把spin\_loop无限循环改为关机(sbi.legacy.shutdown或者srst.system\_reset with RESET\_TYPE\_SHUTDOWN),这样panic就可以让qemu退出。
			\end{block}
			\item TicketMutex[kernel-sync/ticket.rs]
			\begin{block}{}
				对self.next\_ticket的操作缺少内存屏障,既然lock中fetch\_add和is\_locked中的load都是Ordering::Relaxed,
				那么某些cpu在调用is\_locked时,可能会取到next\_ticket的陈旧值。
			\end{block}
		\end{enumerate}
	\end{frame}

	\begin{frame}
		\frametitle{关于GP寄存器}
		没有发现zCore内核里面用GP寄存器relaxing机制,建议可以参照linux kernel用一下这个riscv特色。
		linker.ld中.data和.bss段之间定义一个\_\_global\_pointer\$
		然后入口处启用paging前后各初始化一次gp寄存器
		此外,外部crate trapframe在异常入口位置也要加一下,就在保存S态系统寄存器之后的位置
		\begin{quote}
			\# save sp, sstatus, sepc\\
			… …\\
			STORE\_SP t2, 33         \# save sepc\\
			\vspace {8pt} 
			/* Load the global pointer */\\
			.option push\\
			.option norelax\\
			la gp, \_\_global\_pointer\$\\
			.option pop\\
			\vspace {8pt} 
			andi t1, t1, 1 << 8     \# sstatus.SPP = 1\\
			beqz t1, end\_trap\_from\_user\\
			end\_trap\_from\_kernel:
		\end{quote}
	\end{frame}

	\section{Zircon架构与实现参考}
	
	\begin{frame}
		\begin{center}
			{\LARGE 参照Zircon对zCore调整补充的建议\\}
			\bigskip\bigskip
			{\large Zircon(arm64 on generic-arm) - 10月8日版本}
		\end{center}
	\end{frame}

	\subsection{Zircon总体流程}

	\begin{frame}
		\frametitle{Zircon总体流程}
		ARM64体系结构 - generic-arm平台 - 10月8日版本
	\end{frame}

	\subsection{入口\_start}

	\begin{frame}
		\frametitle{1 入口\_start【kernel/arch/arm64/start.S】}
		\begin{enumerate}
			\item 根据cpuid判断当前cpu身份,0是primary cpu,其它是secondary
			\item primary cpu负责启动工作,所以需要保存引导信息,包括:\\
			handoff地址:上一阶段bootloader传递参数和数据块的开始地址;
			内核入口\_start地址:kernel image的物理地址;
			EL模式:Arm CPU运行级别;
			\item 记录第一条指令的ticks计数作为时间戳\\
			Zircon在各处设置时间戳,度量关键阶段的启动时间。
			\begin{block}{}
				在riscv ISA,我们通过SBI\_PMU\_HW\_CPU\_CYCLES实现。
			\end{block}
			\item 确保当前EL是1,即OS级别,否则切换到OS级
			\begin{block}{}
				zCore也应该检查一下是否S模式,如果意外当前是M模式,要定一个处置策略。
			\end{block}
			\item 启用icache和dcache:Riscv ISA不涉及
			\item 把secondary cpus送去等待,等待primary cpu完成初始化再唤醒它们
			\item 清零BSS
			\item 为primary cpu设置内核栈,该栈在BBS段,长度两个页面
			\begin{block}{}
				如前面提到,zCore仿照该方式,预先只为primary准备栈,后面再按照CPU总数量,按需分配栈
			\end{block}
		\end{enumerate}
	\end{frame}

	\begin{frame}
		\frametitle{1 入口\_start【kernel/arch/arm64/start.S】}
		\begin{enumerate}\setcounter{enumi}{8}
			\item 初始化早期内存分配器boot allocator\\
			两个成员start和end标记已分配区域的开始和结束地址。
			目前还没有分配,所以start/end都设置成kernel image的结束地址\_end,这个也是传统上堆的开始处。
			\item 初始化根页目录,在虚拟地址空间中线性偏移映射两个地址区间\\
			(1)整个物理地址空间映射到虚拟地址空间的固定位置\\
			参数KERNEL\_ASPACE\_BASE指定虚拟空间位置,0~ARCH\_PHYSMAP\_SIZE指定被映射的物理空间范围。在Arm64 ISA,ARCH\_PHYSMAP\_SIZE是512G,所以实质上是整个物理空间。
			映射区域标识为MMU\_PTE\_KERNEL\_DATA\_FLAGS,即只允许读写。
			(2)内核本身占据的物理空间映射到虚拟地址空间的固定位置(由参数KERNEL\_BASE决定)\\
			映射区域标识为MMU\_PTE\_KERNEL\_RWX\_FLAGS,因为包含代码和数据,所以允许读写执行。
			(3)由此在Boot阶段将形成如下的映射布局。内核image在虚拟空间中会被同时映射到两个区间,但是访问权限不同。
			\begin{block}{}
				可以借鉴zircon这个办法:规划时先定分页模式(默认4级页表,相当于RISCV的SV48) ,再根据模式定参数(KERNEL\_BASE和KERNEL\_ASPACE\_BASE),划分出各区域;
				实现中通过kernel vmar管理该布局。
			\end{block}
		\end{enumerate}
	\end{frame}

	\begin{frame}
		\frametitle{1 入口\_start【kernel/arch/arm64/start.S】}
		\begin{enumerate}\setcounter{enumi}{10}
			\item 准备跳板 页根目录,准备启用paging。这个机制与zCore目前一致。
			\item 启用paging\\
			Prmiary hart本身跳转到虚拟空间;然后通知secondary harts,跳板准备完毕,所有secondary harts启用分页后,又会进入spin状态,等待下一个唤醒信号。
			\item 由于空间的切换,内核栈必须重置。\\
			设置线程指针寄存器tpidr\_el1:当前的执行流相当于第一个kthread,zircon定义了一个线程对象表示它,然后把指针存到寄存器tpidr\_el1 。
			将来运行过程中,主要通过这个指针寄存器的切换实现任务的切换。Linux kernel for RISCV也有对应的实现机制:
			寄存器TP在启动阶段指向idle task,TP始终指向当前任务,切换任务时切换TP。
			设置percpu管理数组:现在只需要为Primary Hart设置,对应于percpu的0号元素。
			\item 再次采样计数器ticks,减去开始时的ticks,度量第一阶段的启动速度。
			\item 准备跳转lk\_main,进入C世界\\
			以上只是primary cpu的执行流程,对于arm64,开始只有一个ID为0的cpu作为primary启动,随后必须由它去启动其它cpus。
		\end{enumerate}
	\end{frame}

	\subsection{C入口lk\_main}
	
	\begin{frame}
		\frametitle{2 C入口lk\_main}
		\begin{enumerate}
			\item 初始化线程上下文\\
			初始化线程链表
			初始化percpu的boot\_cpu信息
			构造本cpu的idle线程,加入线程列表,idle作为cpu的保底调度线程
			\item 早期日志初始化
			\item 动态Hook:从Earliest ~ Arch\_early\\
			只有一个Hook,init\_build\_id用于生成版本和编译ID
			\item 体系结构早期初始化:Arch\_early\_init\\
			(1)Cpu早期初始化
			(2)Percpu早期初始化
			(3)设置异常向量入口
			(4)保存cpu features
			(5)启用中断
			(6)MMU早期初始化:初始化ASID分配器
			\item 动态Hook:从Arch\_early ~ Platform\_early
			\item 保存handoff地址,这个是上阶段bootloader传递数据的入口
			\item 平台早期初始化:platform\_early\_init\\
			(1)设置一个系统保留内存区域的链表rev\_list,按照地址从小到大排列。
			先把kernel image本身占据的区间加入rev\_list。\\
			(2)从PhysHandoff发现所有的内存区域,分别处理
			\begin{block}{}
				zCore中,我们通过解析DTB,执行同样的逻辑。
			\end{block}
		\end{enumerate}
	\end{frame}

	\begin{frame}
		\frametitle{2 C入口lk\_main}
		\begin{enumerate}\setcounter{enumi}{8}
			\item 处理NVRAM:该区域主要用作crashlog和tracelog,加入rev\_list。
			\item 处理Common内存区域。包括三类:RAM/Peripheral/Reserve。
			\item 处理RAM:加入mem\_arena数组,上限16个。\\
			Zircon对物理内存的概念是三级:Node、Arena和Page。
			Node可以由一至多个Arena构成,每个Arena内部地址是连续的,内部特性是一致的。
			每个Arena在自己管理区域的顶端位置,设置一个page\_array\_数组,每元素对应维护本区域的页面属性。
			\begin{block}{}
				zCore目标平台采取NUMA的可能性不大,可忽略Node;但是仍可能存在多个Arena,所以保留mem\_arena数组,上限数量16应该是充足的。
			\end{block}
			\item Peripheral:映射到Kernel image之前的区域中,即KERNEL\_BASE之前。上限4个。
			\item Reserve:其它系统保留用途的内存区域,加入rev\_list。
			\item 解析命令行,处理针对debuglog和串口的参数。查看是否需要执行物理内存扫描检查。
			\begin{block}{}
				zCore解析dtb,chosen.bootargs对应命令行,chosen.stdout-path对应默认输出设备,进而解析对应uart设备的参数。
			\end{block}
			\item 把ram disk占据区间加入rev\_list
			\item 如果memory limit存在,对所有的arena区域执行limit交集处理。
			\item 对所有的arena区域与rev\_list执行交集检查,裁剪排除被保留的部分。
			\item 所有保留区域的Page设置为wired,表示被系统保留
		\end{enumerate}
	\end{frame}

	\begin{frame}
		\frametitle{2 C入口lk\_main}
		\begin{enumerate}\setcounter{enumi}{8}
			\item 系统驱动初始化:包括Uart、gic、watchdog、timer、psci,power
			\item 动态Hook:从Platform\_early ~ Arch\_prevm。初始化timer和全局随机数种子发生器。
			\item 虚拟内存初始化准备(体系结构相关):arch\_prevm\_init
			\item 动态Hook:从Arch\_prevm ~platform\_prevm
			\item 虚拟内存初始化准备(平台相关):platform\_prevm\_init\\
			(1)内核空间惯例初始化:以根vmar为树根的树形管理,每一个结点代表位于父节点区间内的子区间
			(2)初始化physmap vmar:对应于前面布局中的physmap区域
			\item 动态Hook:从platform\_prevm ~ vm\_preheap
			\item 堆初始化前准备vm\_init\_preheap\\
			(1)内存地址空间初始化\\
			建立vmar的树表示kernel aspace\\
			root\_vmar树根下,建立三个子vmar(对应前面的虚拟内存布局):\\
			Vmar\_physmap:物理内存空间直接映射(按照512G,可认为就是全部物理内存空间)\\
			Vmar\_image:内核image 自身占据区域\\
			Vmar\_heap:从\&\_end开始的堆区域\\
			(2)Vmar正式加入树(Activate vmar):\\
			前置检查vmar在parent的范围内,且与当前所有的兄弟不冲突。\\
			(3)分配vmar:在parent区间中寻找一块未使用的和符合对齐要求的区间\\
			(4)遍历boot allocator,把目前已经分配出去Page标记为wired进行保留\\
			(5)分配一页作为只读的Zero Page,供整个系统使用\\
			(6)匿名页面请求器初始化
		\end{enumerate}
	\end{frame}

	\begin{frame}
		\frametitle{2 C入口lk\_main}
		\begin{enumerate}\setcounter{enumi}{10}
			\item 动态Hook:从vm\_preheap ~ heap:仅一个HOOK,初始化内存越界检查机制asan
			\item 初始化堆heap\_init\\
			(1)初始化分配器cmpct,每次heap\_grow增加256K空间\\
			(2)扩展堆heap\_grow:优先使用缓存,没有缓存时从pmm申请\\
			* Cmpct相当于Linux kernel的buddy system;Pmm的Node和Area对应于pgdat和zone
			\item VirtualAllocat基于堆实现\\
			(1)通过BitmapAlloc从位图中查找并标记分配\\
			(2)从cmpct中申请物理Page\\
			(3)把物理Page映射到虚拟空间,返回地址
			\item 动态Hook:heap ~ vm,包括如下几个hook
			\item 资源系统初始化Hook\\
			mmio资源:64位空间for mmio\\
			中断资源:全局中断控制器gic,irq资源\\
			系统资源:ioport,root,smc,system,count
			\item Version打印Hook
			\item 控制台初始化Hook:即对Console参数的处理和分配缓冲区
			\item 虚拟内存初始化vm\_init
			\item 地址空间空隙保护\\
			physmap虚拟空间中可能存在gap区域,对它们设置特殊属性,禁止访问
			\item 对所有直接映射的Arenas区域进行保护:所有的Arena区域都不可执行
			\item 针对内核image各段(代码段、数据段)的特性进行保护
		\end{enumerate}
	\end{frame}

	\begin{frame}
		\frametitle{2 C入口lk\_main}
		\begin{enumerate}\setcounter{enumi}{20}
			\item 对vmar\_physmap执行保护策略
			\item 设置threshold(即水位),当前可分配空间低于该界限时自动从pmm填充
			\item 动态Hook:vm ~ topology\\
			(1)从Handoff解析topology信息,填充管理cpu的层次结构\\
			如果zCore目标平台不涉及NUMA,只要初始化一个cpu管理数组\\
			(2)如果存在periph vmar,对其设置保护策略\\
			\item NUMA初始化:略过
			\item 动态Hook:topology ~ kernel
			\item 内核初始化:kernel\_init
			\item 动态Hook:kernel ~ threading\\
			(1)初始化pmm前端页面缓存\\
			为每个cpu设置一定数量的Pages作为缓存,避免每次都要从pmm中申请\\
			(2)缓冲区链初始化buffer\_chain\_cache\_init\\
			Zircon的buffer类似于linux kernel的buffer,可以跨多个Page,但是它们主要的服务对象是通道对象Channel。所以这个链的作用相当于Slab for channel。\\
			(3)端口缓冲区初始化port\_observer\_cache\_init\\
			为端口的observer和packet的分配提供缓冲页面,相当于Slab for port。\\
			(4)代码覆盖率检测机制初始化InitSancov\\
			(5)写时拷贝缓冲区初始化InitCowPages:预留一部分Pages专用于该目的\\
			(6)初始化Percpu数组\\
			每个cpu按照逻辑ID在percpu数组中对应一个管理对象,其中0号时Primary cpu。\\
			启动早期已经初始化Primary cpu,现在完成其它secondary cpus。
			\item 启动bootstrap2线程,在线程中开启下一阶段的启动\\
			创建启动线程见3;bootstrap2的执行逻辑见4。
		\end{enumerate}
	\end{frame}

	\subsection{启动内核线程,准备第二阶段启动}
	
	\begin{frame}
		\frametitle{3 启动内核线程,准备第二阶段启动}
		\begin{enumerate}
			\item 线程结构Thread\\
			(1)调度器scheduler\_state:调度相关,例如优先级\\
			(2)等待队列wait\_queue\_state:在等待队列中的接入点\\
			(3)任务task\_state:任务相关,记录执行入口entry和启动参数arg\\
			(4)抢占preemption\_state:抢占相关\\
			\item 创建线程实例CreateEtc\\
			(1)分配一个线程实例\\
			(2)在task\_state中记录entry 和 arg\\
			(3)通过调度器初始化该线程,作用在scheduler\_state上\\
			(4)设置线程栈\\
			(4.1)从内核空间的root\_vmar中分配子vmar,大小根据本体系结构的默认栈大小值\\
			(4.2)分配COW页面,返回封装它们的VMO对象\\
			(4.3) 为VMO对象在虚拟空间建立映射,然后触发fault in,完成内容的填充\\
			(4.4)根据体系结构的要求,初始化栈帧\\
			(4.5)把线程推入thread list\\
			\item 脱离父线程Detach
			标记本线程detached
			\item 恢复(或启动)线程\\
			主要通过Scheduler::Unblock解除线程的阻塞状态,实现恢复或者启动\\
			(1)取时间戳\\
			(2)取目标CPU,然后根据cpu逻辑号找到对应的scheduler实例\\
			(3)设置线程状态为ready,推入scheduler的调度队列\\
			(4)请求reschedule,促使该线程被调度到
		\end{enumerate}
	\end{frame}

	\subsection{在内核线程中,执行第二阶段启动}
	
	\begin{frame}
		\frametitle{4 第二阶段启动bootstrap2(在内核线程中执行)}
		\begin{enumerate}
			\item 设置当前线程的CPU亲和性(绑定primary cpu)\\
			当前线性继续代表Primary CPU执行初始化,因此必须绑定在primary cpu上。
			\item 动态Hook:Threading ~ Arch
			\item 对象初始化Hook\\
			(1)为Handle准备存储区域:分配一部分Pages作为作为分配handle的缓存\\
			(2)初始化Executor:系统调用与底层内核资源的中间层\\
			(3)创建root\_job以及它的Handle
			\item 全局随机数发生器Hook:建立并保证线程安全
			\item 死锁检测
			\item 定时器初始化:启用时钟中断,设置时钟stream模式
			\item DPC初始化:Deffered Procedure Calls\\
			每个cpu的percpu维护一个DPC队列,保存延迟调用的任务。
			通常用于减轻中断处理的任务,从中断上下文转到普通上下文中处理。
			\item 体系结构初始化arch\_init
			\item 初始化percpu关于中断的部分,然后让本cpu上线,并启用IPI中断
			\item 为每个secondary cpu创建IDLE Thread(Primary 的之前已经建立)
			\item 动态Hook:Arch ~ Platform
			\item 性能监控Hook:perfmon\_init
			\item 初始化debug端口Hook:Debug Port用于从外部连接和调试Soc
			\item 平台初始化platform\_init
			\item 注册CPU逻辑ID与物理ID的映射表
			\item 为所有secondary cpu准备内核栈
			\item 唤醒启动所有的secondary cpus
			\item 启动一个线程,用于检查是否所有的cpu都已经成功启动\\
			具体实现方法:等待mp\_state.online\_cpus位图的所有有效位都已经置一 (每个CPU启动成功后会在对应位上置一)
		\end{enumerate}
	\end{frame}

	\begin{frame}
		\frametitle{4 第二阶段启动bootstrap2(在内核线程中执行)}
		\begin{enumerate}
			\item 内核驱动后期初始化:\\
			(1)Uart的后期初始化\\
			(2)Gic、hdmi、rng、watchdog后期初始化
			\item 动态Hook:Platform ~ Arch\_late\\
			(1)调试日志hook:启动debuglog线程\\
			(2)全局随机数发生器Hook:生成种子\\
			(3)时间段度量机制初始化\\
			(4)所有cpu启动计数
			\item 体系结构后期初始化Arch\_late\_init\\
			标记当上下文切换时,分支预测器是否需要刷新
			\item 动态Hook:Arch\_late ~ Last:\\
			现在仅有一个Hook:User\\
			(1)显式内核启动后空闲的页面数量\\
			(2)初始化计数器,记录zircon 从启动开始到完成的总时间。\\
			(3)初始化Shell:启动主控台\\
			(4)初始化资源过滤器:对资源进行保护\\
			目前把RAM加入过滤器,不允许userspace直接访问;但是允许直接访问reserve的RAM区域,因为有可能用户进程在其中有mmio区间。\\
			(5)初始化Trace:ktrace\_init\\
			(6)启动线程处理全局随机数发生器\\
			(7)用户态初始化userboot\_init\\
			以下开启下一阶段,将从内核态转入用户态。流程概述\\
			(7.1)Userboot\_init先做准备工作(详见5) ,最后启动一个线程做后续工作\\
			(7.2)在线程中,通过vdso启动第一个用户态程序UserBoot(详见6)\\
			(7.3)UserBoot依然是系统的一部分,它负责启动第一个真正有用的用户程序(zCore目前就是sh)
		\end{enumerate}
	\end{frame}

	\subsection{准备启动用户态程序userboot\_init}
	
	\begin{frame}
		\frametitle{5 准备启动用户态程序userboot\_init}
		\begin{enumerate}
			\item 构造消息包packet,作为句柄数组的载体,该消息将来会通过通道channel传给进程
			\item 通过工厂ProcessDispatcher来创建进程process\\
			(1)在root\_job下构建一个process\\
			(2)创建process工厂的句柄\\
			(3)创建vmar工厂的句柄\\
			(4)返回上述句柄
			\item 创建进程对象process和地址区间对象vmar的句柄,填充到句柄数组中
			\item 分配资源句柄,填充到句柄数组中\\
			主要资源包括root, mmio, irq, smc, system以及root\_job,这些资源的句柄将被传递给process供它使用。
			\item 创建和获取各vdso和vmo对象\\
			vDSO都是ELF格式的对象,类似于DSO。\\
			区别是vDSO是kernel直接在地址空间中分配并填充构成的。\\
			(1)通过申请空间和初始化vDSO基础结构\\
			(2)创建常量,填充到vDSO\\
			(3)Patch vDSO,加入必要的系统调用\\
			(4)通过bootstrap\_vmos,获取各种参数填充vDSO\\
			(5)把vDSO的句柄和ZBI的句柄都填充到句柄数组
		\end{enumerate}
	\end{frame}

	\begin{frame}
		\frametitle{5 准备启动用户态程序userboot\_init}
		\begin{enumerate}
			\item 构造channel,包括两端,内核端和进程端\\
			(1)通过channel的内核端,把前面创建的消息写入channel\\
			(2)channel用户端会在process启动时交给它\\
			进程一旦启动,就能够从channel中读消息,取出句柄数组
			\item 把userboot\_image映射到vmar管理的空间中\\
			(1)Image包含userboot本身和紧跟的vdso,它们作为进程root\_vmar的子vmar被分别映射到虚拟空间\\
			(2)返回两个关键信息,userboot的执行入口entry和vdso\_base地址\\
			\item 映射栈
			\item 创建用户进程中的首个线程\\
			类似于Process,通过ThreadDispatcher,即线程工厂创建线程。\\
			但是该线程的启动比较特殊,在切换运行状态的同时,完成线程启动。\\
			(1)伪造栈帧现场,做好返回用户态的准备\\
			关键是两个参数:线程entry作为epc,线程栈作为ustack pointer。\\
			(2)处理待决信号(pending signal):这里正是处理信号的时机\\
			(3)切换到用户态arch\_enter\_uspace(iframe)\\
			禁止中断,基于前面伪造的返回现场,通过eret指令返回用户态执行entry,此后就是在userspace执行逻辑。
			\item 已经完成从内核态到用户态的切换,执行第一个用户态程序UserBoot(详见6)
		\end{enumerate}
	\end{frame}

	\subsection{第一个用户态程序userboot}

	\begin{frame}
		\frametitle{6 第一个用户态程序userboot}
		\begin{enumerate}
			\item 入口\_start(zx\_handle\_t arg)\\
			arg参数就是由内核创建的channel的用户端
			\item 从channel读出消息,取出所有句柄
			\item 通过对应句柄,获取vmar对象和process对象
			\item 通过句柄从ZBI中取得bootfs,这是一个简单的文件系统
			\item 从ZBI中取得命令行参数
			\item 加载真正的用户态进程(init/sh等),移交执行权,此刻系统启动阶段完成
			\item 本进程正常退出zx\_process\_exit(0)
		\end{enumerate}
	\end{frame}

	\subsection{Secondary CPUs初始化与启动过程总结}

	\begin{frame}
		\frametitle{7 Secondary CPUs初始化与启动过程总结}
		\begin{enumerate}
			\item 确保在E1态运行
			\item 启用icache/dcache/ucache
			\item 取得跳板页目录地址和正式根目录地址
			\item 等待prime的通知,启用mmu分页
			\item 等待secondary boot信号,开始正式运行
			\item 设置栈\\
			(1) primary cpu在发出secondary boot信号前,按照cpu总数为其它所有的cpu准备好栈空间(地址连续)\\
			(2) 这里secondary cpu根据自己的逻辑ID定位对应的栈\\
			\item 进入secondary入口执行arm64\_secondary\_entry\\
			(1)初始化percpu结构\\
			(2)设置本cpu异常入口\\
			(3)获取本cpu features\\
			(4)启用中断\\
			(5)等待secondaries\_released信号\\
			该信号在primary\_cpu在arch\_init中为所有secondary cpus创建IDLE thread之后。
			\item 构造Thread,继续执行初始化任务
		\end{enumerate}
	\end{frame}

	\begin{frame}
		\frametitle{7 Secondary CPUs初始化与启动过程总结}
		\begin{enumerate}
			\item 执行动态hook,从Earliest ~ Threading\\
			注意:各种hook在注册时,指定了适用的目标。\\
			大多数HOOK的标识是LK\_INIT\_FLAG\_PRIMARY\_CPU,即只对primary cpu有效。\\
			只有少数标记为LK\_INIT\_FLAG\_SECONDARY\_CPUS或LK\_INIT\_FLAG\_ALL\_CPUS的Hook会在这被执行。
			\item 初始化cpu的根中断管理器
			\item 进入线程执行最后阶段的初始化\\
			(1)Percpu的后期初始化:针对体系结构的特殊要求\\
			(2)动态Hook:Threading~Last\\
			(3)死锁检测
			\item secondary cpu执行初始化任务的线程退出\\
			(1)从调度器中删除本线程\\
			(2)执行Thread::Current::Exit()退出\\
			(2.1)标记本线程状态death\\
			(2.2)设置返回码\\
			(2.3)清理资源\\
			如果是标记了Detached的线程,自己清理资源,然后直接从全局线程list中删除;\\
			否则需要通知parent thread清理资源(Parent正在阻塞等待本线程退出)。\\
			(2.4)执行重调度RescheduleInternal,切换到其它线程\\
			Secondary cpu在启动阶段,没有其它的线程或进程在执行,所以退回到本cpu的IDLE Thread。
		\end{enumerate}
	\end{frame}

	\subsection{Prime CPU成为IDLE}

	\begin{frame}
		\frametitle{8 Prime CPU成为IDLE}
		\begin{enumerate}
			\item 设置名称“idle [cpuid]”
			\item 标记本线程是IDLE
			\item 从调度器中删除,不参与调度,idle就是垫底的
			\item 线程设置running状态
			\item 设置当前CPU是active状态,表示活动中
			\item 设置pending for preemption状态,确保重调度reschedule可以执行
			\item 启用preempt抢占机制
			\item 启用本CPU的中断
			\item 进入无限循环等待状态
		\end{enumerate}
	\end{frame}

	%!!!!!!!!!!!!!!!!!!!!!!!!!!!!!!!!!!!!!!!!!!!!!!!!!!!!!!!!!!!!!!!
	%	Demo
	%!!!!!!!!!!!!!!!!!!!!!!!!!!!!!!!!!!!!!!!!!!!!!!!!!!!!!!!!!!!!!!!
	
	\begin{frame}
		\frametitle{X XXX}
		\begin{enumerate}
			\item XXXX
			\begin{block}{}
				XXXX
			\end{block}
			\item XXXX
		\end{enumerate}
	\end{frame}

	\section{Text Examples}

	\subsection{Paragraphs and Lists}

	\begin{frame}
	\frametitle{Paragraphs of Text}

	Sed iaculis \alert{dapibus gravida}. Morbi sed tortor erat, nec interdum arcu. Sed id lorem lectus. Quisque viverra augue id sem ornare non aliquam nibh tristique. Aenean in ligula nisl. Nulla sed tellus ipsum. Donec vestibulum ligula non lorem vulputate fermentum accumsan neque mollis.

	\bigskip % Vertical whitespace

	\begin{quote}
		Sed diam enim, sagittis nec condimentum sit amet, ullamcorper sit amet libero. Aliquam vel dui orci, a porta odio.\\
		--- Someone, somewhere\ldots
	\end{quote}

	\bigskip % Vertical whitespace

	Nullam id suscipit ipsum. Aenean lobortis commodo sem, ut commodo leo gravida vitae. Pellentesque vehicula ante iaculis arcu pretium rutrum eget sit amet purus. Integer ornare nulla quis neque ultrices lobortis.
	\end{frame}

	\subsection{Blocks}
	
	\begin{frame}
		\frametitle{Blocks of Highlighted Text}
		
		\begin{block}{Block Title}
			Lorem ipsum dolor sit amet, consectetur adipiscing elit. Integer lectus nisl, ultricies in feugiat rutrum, porttitor sit amet augue.
		\end{block}
		
		\begin{exampleblock}{Example Block Title}
			Aliquam ut tortor mauris. Sed volutpat ante purus, quis accumsan.
		\end{exampleblock}
		
		\begin{alertblock}{Alert Block Title}
			Pellentesque sed tellus purus. Class aptent taciti sociosqu ad litora torquent per conubia nostra, per inceptos himenaeos.
		\end{alertblock}
		
		\begin{block}{} % Block without title
			Suspendisse tincidunt sagittis gravida. Curabitur condimentum, enim sed venenatis rutrum, ipsum neque consectetur orci.
		\end{block}
	\end{frame}

	\subsection{Columns}
	
	\begin{frame}
		\frametitle{Multiple Columns}
		\framesubtitle{Subtitle} % Optional subtitle
		
		\begin{columns}[c] % The "c" option specifies centered vertical alignment while the "t" option is used for top vertical alignment
			\begin{column}{0.45\textwidth} % Left column width
				\textbf{Heading}
				\begin{enumerate}
					\item Statement
					\item Explanation
					\item Example
				\end{enumerate}
			\end{column}
			\begin{column}{0.5\textwidth} % Right column width
				Lorem ipsum dolor sit amet, consectetur adipiscing elit. Integer lectus nisl, ultricies in feugiat rutrum, porttitor sit amet augue. Aliquam ut tortor mauris. Sed volutpat ante purus, quis accumsan dolor.
			\end{column}
		\end{columns}
	\end{frame}

	\section{Table and Figure Examples}
	
	\subsection{Table}
	
	\begin{frame}
		\frametitle{Table}
		\framesubtitle{Subtitle} % Optional subtitle
		
		\begin{table}
			\begin{tabular}{l l l}
				\toprule
				\textbf{Treatments} & \textbf{Response 1} & \textbf{Response 2}\\
				\midrule
				Treatment 1 & 0.0003262 & 0.562 \\
				Treatment 2 & 0.0015681 & 0.910 \\
				Treatment 3 & 0.0009271 & 0.296 \\
				\bottomrule
			\end{tabular}
			\caption{Table caption}
		\end{table}
	\end{frame}

	\subsection{Figure}
	
	\begin{frame}
		\frametitle{Figure}
		
		\begin{figure}
			\includegraphics[width=0.8\linewidth]{creodocs_logo.pdf}
			\caption{Creodocs logo.}
		\end{figure}
	\end{frame}

	\section{Mathematics}
	
	\begin{frame}
		\frametitle{Definitions \& Examples}
		
		\begin{definition}
			A \alert{prime number} is a number that has exactly two divisors.
		\end{definition}
		
		\smallskip % Vertical whitespace
		
		\begin{example}
			\begin{itemize}
				\item 2 is prime (two divisors: 1 and 2).
				\item 3 is prime (two divisors: 1 and 3).
				\item 4 is not prime (\alert{three} divisors: 1, 2, and 4).
			\end{itemize}
		\end{example}
		
		\smallskip % Vertical whitespace
		
		You can also use the \texttt{theorem}, \texttt{lemma}, \texttt{proof} and \texttt{corollary} environments.
	\end{frame}

	\section{Referencing}
	
	\begin{frame}
		\frametitle{Citing References}
		
		An example of the \texttt{\textbackslash cite} command to cite within the presentation:
		
		\bigskip % Vertical whitespace
		
		This statement requires citation \cite{p1,p2}.
	\end{frame}
	
	%------------------------------------------------
	
	\begin{frame} % Use [allowframebreaks] to allow automatic splitting across slides if the content is too long
		\frametitle{References}
		
		\begin{thebibliography}{99} % Beamer does not support BibTeX so references must be inserted manually as below, you may need to use multiple columns and/or reduce the font size further if you have many references
			\footnotesize % Reduce the font size in the bibliography
			
			\bibitem[Smith, 2022]{p1}
			John Smith (2022)
			\newblock Publication title
			\newblock \emph{Journal Name} 12(3), 45 -- 678.
			
			\bibitem[Kennedy, 2023]{p2}
			Annabelle Kennedy (2023)
			\newblock Publication title
			\newblock \emph{Journal Name} 12(3), 45 -- 678.
		\end{thebibliography}
	\end{frame}

	\begin{frame}
		\frametitle{Acknowledgements}
		
		\begin{columns}[t] % The "c" option specifies centered vertical alignment while the "t" option is used for top vertical alignment
			\begin{column}{0.45\textwidth} % Left column width
				\textbf{Smith Lab}
				\begin{itemize}
					\item Alice Smith
					\item Devon Brown
				\end{itemize}
				\textbf{Cook Lab}
				\begin{itemize}
					\item Margaret
					\item Jennifer
					\item Yuan
				\end{itemize}
			\end{column}		
			\begin{column}{0.5\textwidth} % Right column width
				\textbf{Funding}
				\begin{itemize}
					\item British Royal Navy
					\item Norwegian Government
				\end{itemize}
			\end{column}
		\end{columns}
	\end{frame}

	\begin{frame}[plain] % The optional argument 'plain' hides the headline and footline
		\begin{center}
			{\Huge The End}
			
			\bigskip\bigskip % Vertical whitespace
			
			{\LARGE Questions? Comments?}
		\end{center}
	\end{frame}

\end{document}